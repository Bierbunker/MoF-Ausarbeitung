\documentclass[a4paper, 11pt, ngerman, parskip=half-]{scrartcl}

\title{MoF - Fragenkatalog}
\subtitle{\href{https://github.com/Bierbunker/MoF-Ausarbeitung}{\underline{Aktuelle Version (Link)}}}
\date{\today}

\usepackage[margin=1in]{geometry}

\usepackage[utf8]{inputenc}
\usepackage[T1]{fontenc}
\usepackage{lmodern}
\usepackage{babel}
\usepackage{csquotes}
\usepackage{xurl}

\usepackage{amsmath, amssymb, amstext, mathtools}
\usepackage{icomma}
\usepackage[locale=DE]{siunitx}
\usepackage{physics}
\usepackage{derivative}  % overrides derivatives of package physics!

\usepackage{pdfpages}
\usepackage{lastpage}
\usepackage{graphicx}
\usepackage{float}

\usepackage{xcolor}
\usepackage[hidelinks,colorlinks]{hyperref}
%Colorlinks setup:
\hypersetup{
    colorlinks = false,
    linkbordercolor = {white}
}

\usepackage[headsepline]{scrlayer-scrpage}
\pagestyle{scrheadings}
\setkomafont{pageheadfoot}{\normalfont}
\ihead{\hyperlink{Fehler}{\color{cyan}Fehlermeldungen}}
\chead{MoF Fragenkatalog}
\ohead{\today}
\cfoot{\pagemark{} / \pageref*{LastPage}}

\usepackage{caption, subcaption}
\captionsetup[table]{name=Tabelle}
\captionsetup[figure]{name=Abbildung}
\captionsetup{format=plain, font=small, labelfont=bf, justification=centering}

% Line break after paragraph
\newcommand{\myparagraph}[1]{\paragraph{#1}\mbox{}\\}

% numerical aperture
\newcommand{\NA}{\ensuremath{\mathit{NA}}}


% Question command
\newcounter{question}
\newcommand{\question}[1]{\stepcounter{question}\paragraph{Frage \thequestion: #1}~}



\begin{document}

\maketitle

\newpage

\tableofcontents

\newpage
\section*{Ähnliche Fragen vermutlich (GPT-4)}

Einige Fragen sind ähnlich oder decken ähnliche Themen ab. Hier sind die entsprechenden Paare:

Fragen \ref{q:1} und \ref{q:13} diskutieren die Austauschenergie beim H2 Molekül und ihre Bedeutung für die chemische Bindung.

Fragen \ref{q:2} und \ref{q:14}, \ref{q:21}, \ref{q:32}, \ref{q:40} und \ref{q:55} fragen nach Informationen, die aus dem Rotations-Schwingungsspektrum von Molekülen abgeleitet werden können.

Fragen \ref{q:3} und \ref{q:15} behandeln elektrische Rotations-Schwingungsübergänge und die Entstehung von Schwingungsbanden.

Fragen \ref{q:4} und \ref{q:16}, \ref{q:27} und \ref{q:45}, \ref{q:50} und \ref{q:54} beziehen sich auf den Zusammenhang zwischen der Gitterebene und einem Vektor im reziproken Raum.

Fragen \ref{q:5} und \ref{q:17}, \ref{q:47} und \ref{q:59} beziehen sich auf das Auftreten von Energiebandlücken mit Hilfe des Models der fast freien Elektronen.

Fragen \ref{q:6} und \ref{q:18} behandeln, wie experimentell die Dispersionsrelationskurve von Phononen ermittelt werden kann.

Fragen \ref{q:7} und \ref{q:19}, \ref{q:38}, \ref{q:51} und \ref{q:57} diskutieren die Wärmekapazität sowohl in der klassischen als auch in der quantenmechanischen Betrachtung.

Fragen \ref{q:8} und \ref{q:20} beziehen sich auf das Bloch-Theorem eines Elektrons im harmonischen Potential.

Fragen \ref{q:9} und \ref{q:56} behandeln den Paulschen Paramagnetismus.

Fragen \ref{q:24} und \ref{q:37} beziehen sich auf die Hybridisierung in mehratomigen Molekülen.

Fragen \ref{q:25} und \ref{q:35}, \ref{q:44} behandeln die Laue'sche Beugungsbedingung und die Ewald-Konstruktion.

Fragen \ref{q:26} und \ref{q:36} diskutieren den Unterschied zwischen einem fcc-Gitter und einer hcp-Struktur.

Fragen \ref{q:28} und \ref{q:39}, \ref{q:48} und \ref{q:58} vergleichen die Einstein- und Debye-Modelle der spezifischen Wärme.

Fragen \ref{q:30} und \ref{q:49} erläutern den Atomfaktor und den Strukturfaktor bei der Röntgenbeugung.

Fragen \ref{q:41} und \ref{q:52} behandeln die Lennard-Jones-Potential und Van der Waals Bindung.

Fragen \ref{q:42} und \ref{q:33} behandeln das Konzept des Bravaisgitters und der Wigner-Seitz Zelle.

Fragen \ref{q:43} und \ref{q:31}, \ref{q:34} diskutieren das Franck-Condon Prinzip und die Intensität von Schwingungsbänden in einem Molekülspektrum.

Es ist wichtig zu beachten, dass obwohl diese Fragen ähnliche Themen behandeln, die spezifischen Aspekte oder der Kontext, der in jeder Frage behandelt wird, variieren können. Daher ist es wichtig, jede Frage sorgfältig zu lesen und zu verstehen, bevor man versucht, sie zu beantworten.

\newpage
\section{01. Mai 2015}
\question{Austauschenergie beim H2 Molekül? Wann ist es groß/klein? Wieso ist es für die Bindung wichtig?}
\label{q:1}

\question{Was kann man aus dem Rotations-Schwingungsspektrum für molekulare Größen bestimmen?}
\label{q:2}

\question{Diskutieren Sie elektrische Rotations-Schwingungsübergänge und die Entstehung von Schwingungsbanden.}
\label{q:3}

\question{Was für ein Zusammenhang besteht zwischen der Gitterebene und einem Vektor im reziproken Raum?}
\label{q:4}

\question{Erklären Sie das Auftreten von Energiebandlücken mit Hilfe des Models der fast freien Elektronen.}
\label{q:5}

\question{Wie kann experimentell die Dispersionsrelationskurve von Phononen ermittelt werden?}
\label{q:6}

\question{Diskutieren Sie die Wärmekapazität sowohl in der klassischen als auch in der quantenmechanischen Betrachtung.}
\label{q:7}

\question{Beschreiben Sie das Bloch-Theorem eines Elektrons im harmonischen Potential.}
\label{q:8}

\question{Beschreiben Sie den Paulschen Paramagnetismus.}
\label{q:9}

\question{Potentiat-Bindung im homöopolaren Molekül?}
\label{q:10}

\newpage
\section{05. Mai 2015}

\question{Skizzieren Sie den Verlauf der Potentialkurve für ein homöopolares Molekül. Erklären Sie sie qualitativ.}
\label{q:11}

\question{Erklären Sie die Herkunft der 'Austauschenergie' in H2. Wann ist sie groß, wann klein? Warum ist sie entscheidend für die chemische Bindung?}
\label{q:12}

\question{Welche Informationen über molekulare Kenngrößen können aus Rotations-Schwingungsspektren von Molekülen erhalten werden?}
\label{q:13}

\question{Diskutiere einen elektronischen Übergang in einem 2-Atomigen Molekül und die daraus resultierenden Schwingungsbanden im Spektrum.}
\label{q:14}

\question{Diskutiere den Zusammenhang zwischen Gitterebene im Kristall und Vektoren des reziproken Gitters.}
\label{q:15}

\question{Diskutieren Sie die Wärmekapazität des Gitters im Rahmen des klassischen und der Quantenmechanik.}
\label{q:16}

\question{Wie ermittelt man experimentell eine Phononendispersionskurve?}
\label{q:17}

\question{Erklären Sie das Zustandekommen von Energiebändern in Festkörper im Rahmen der Theorie der fast-freien Elektronen.}
\label{q:18}

\question{Erklären Sie das Bloch'sche Theorem für Elektronen im periodischen Potential.}
\label{q:19}

\question{Diskutieren Sie den Paulschen Paramagnetismus.}
\label{q:20}
\newpage
\section{09. Mai 2012}

\question{Skizzieren der wesentlichen Elemente der Born-Oppenheimer Näherung.}
\label{q:21}

\question{Welche Informationen über molekulare Kerngrößen können aus Rotations-Schwingungsspektren von Molekülen erhalten werden?}
\label{q:22}

\question{Diskutieren das Schwigungs-Rotations Spektrum eines 2-atomigen Moleküls.}
\label{q:23}

\question{Welche Verbesserungen des einfachen MO- oder VB-Ansatzes ermöglichen eine bessere Übereinstimmung mit den experimentellen Werten?}
\label{q:24}

\question{Was versteht man unter dem Franck-Condon Prinzip? Diskutiere die Intensität von Schwingungsbänden in einem Molekülspektrum bei einem elektronischen Übergang an Hand des Franck-Condon Prinzips.}
\label{q:25}

\question{Diskutiere die Laue'sche Beugungsbedingung: anhand Ewald Konstruktion im Rahmen der Bragg'schen Interpretation.}
\label{q:26}

\question{Erläutere den Atomfaktor und den Strukturfaktor bei der Röntgenbeugung.}
\label{q:27}

\question{Skizziere die 1. Brillouin Zone eines ebenen ParaRechteckgitters.}
\label{q:28}

\question{Wodurch unterscheiden sich akustische von optischen Phononen?}
\label{q:29}

\question{Vergleiche die Einstein- und Debye-Modelle der spez. Wärme. Welche Annahme ist im Einstein Modell zu einfach?}
\label{q:30}

\newpage
\section{15. Juni 2015}

\question{Elektronenkonfiguration von $B_2$ ($Z=5$)}
\label{q:31}

\question{Arten von Para-Diamagnetismus}
\label{q:32}

\question{Konzept von Bravais Wiener-Seitz Zelle}
\label{q:33}

\question{Sp sp2 sp3 Bindungen von mehratomigen Molekülen}
\label{q:34}

\question{Debye-Petit (oder so, irgendwas mit spezifischer Wärme von Elektronen)}
\label{q:35}

\question{Atom/Struktur Faktor}
\label{q:36}

\question{Zustandekommen von Energiebändern und Bandlöchern}
\label{q:37}

\question{Zusammenhang zwischen Gitterebene und Vektoren im reziproken Gitter}
\label{q:38}

\question{Statische Abschirmung des Elektronengases}
\label{q:39}

\question{Einstein-Debye Unterschiede und falsche Annahmen}
\label{q:40}

\newpage
\section{16. März 2012}

\question{Diskutieren der unterschiedlichen Überlegungen in der Valenzbindugsnäherung (VB) und in der Molekularorbitalnäherung (MO) zur Beschreibung der Molekülbindung in einem 2-atomigen Molekül}
\label{q:41}

\question{Skizzieren der bindenden und antibindenden Wellenfunktionen in einem homonuklearen 2-atomigen Molekül}
\label{q:42}

\question{Diskutieren des Schwigungs-Rotations Spektrums eines 2-atomigen Moleküls}
\label{q:43}

\question{Diskutiere die chemische Bindung im Li2 Molekül}
\label{q:44}

\question{Was versteht man unter dem Franck-Condon Prinzip? Diskutiere die Intensität von Schwingungsbänden in einem Molekülspektrum bei einem elektronischen Übergang an Hand des Franck-Condon Prinzips.}
\label{q:45}

\question{Wodurch unterscheidet sich ein fcc Gitter von einem hcp?}
\label{q:46}

\question{Erläutere den Atomfaktor und den Strukturfaktor bei der Röntgenbeugung.}
\label{q:47}

\question{Skizziere die 1. Brillouin Zone eines ebenen Parallelogrammgitters.}
\label{q:48}

\question{Wodurch unterscheidet sich die Dispersionsrelation der Phononen eines primitiven kubischen Gitters von jener eines CsCl-Gitters.}
\label{q:49}

\question{Vergleiche die Einstein- und Debye-Modelle der spez. Wärme. Welche Annahme ist im Einstein Modell zu einfach?}
\label{q:50}

\newpage
\section{24. März 2015}

\question{Franck Condon Prinzip \& Intensität der Übergänge}
\label{q:51}

\question{Diskutiere die Elektronenkonfiguration von N2.}
\label{q:52}

\question{Diskutiere die Van der Waals Wechselwirkung und das Lennard-Jones Potential.}
\label{q:53}

\question{Konzept des Bravaisgitters und der Wigner-Seitz Zelle.}
\label{q:54}

\question{Diskutiere den Unterschied zwischen der akustischen und optischen Phononendispersionsrelation bzw. longitudinaler/transversaler, akustischer und optischer Phononen.}
\label{q:55}

\question{Diskutiere das statische Verhalten des freien Elektronengases.}
\label{q:56}

\question{Diskutiere die Einstein- und Debye-Näherung der Wärmekapazität.}
\label{q:57}

\question{Diskutiere den Beitrag der Elektronen zur Wärmekapazität.}
\label{q:58}

\question{Was besagt das Bloch Theorem bzw. die Bloch Funktion?}
\label{q:59}

\question{Welche Arten von Dia- und Paramagnetismus kennen Sie?}
\label{q:60}

\newpage
\section{28. November 2018}
%! Hier gibts schon ein bisschen was hier: https://docs.google.com/document/d/1VvuRhqSy6umcpChijdMtuYw5qB0H7iJZGPXQbSMj6zA/edit
\question{Skizzieren Sie die wesentlichen Elemente der Born-Oppenheimer Näherung.}
\label{q:61}

\question{Diskutieren Sie die sp-, sp2-, sp3-Hybridisierung in mehratomigen Molekülen.}
\label{q:62}

\question{Diskutieren Sie die Laue'sche Beugungsbedingung anhand der Ewald-Konstruktion im Rahmen der Bragg'schen Interpretation.}
\label{q:63}

\question{Wodurch unterscheiden sich ein fcc-Gitter von einer hcp-Struktur?}
\label{q:64}

\question{Diskutieren Sie die Gitterenergie der Ionenkristalle.}
\label{q:65}

\question{Vergleichen Sie die Einstein- und Debye-Modelle der spezifischen Wärme. Welche Annahmen sind in beiden Modellen zu einfach?}
\label{q:66}

\question{Diskutieren Sie das Auftreten einer Energiebandlücke mit Hilfe des Modells der fast freien Elektronen.}
\label{q:67}

\question{Wodurch unterscheidet sich die Dispersionsrelation der Phononen eines primitiven kubischen Gitters von jenem eines CsCl-Gitters.}
\label{q:68}

\question{Erklären Sie die chemische Bindung von $O_2$ (O; $Z=8$).}
\label{q:69}

\newpage
\section{31. Oktober 2013}

\question{Bindungszustände / Elektronenkonfiguration in N2 diskutieren.}
\label{q:70}

\question{Skizzieren Sie die wesentlichen Elemente der Born-Oppenheimer Näherung.}
\label{q:71}

\question{Erklärung von Hybridisierung anhand sp, sp2 und sp3.}
\label{q:72}

\question{Lennard-Jones Potential und Van der Waals Bindung.}
\label{q:73}

\question{Diskutieren Sie die Laue'sche Beugungsbedingung.\
\label{q:74}
\qquad \qquad a) Anhand der Ewald-Konstruktion.\
\qquad \qquad b) Im Rahmen der Bragg'schen Interpretation.}

\question{Skizzieren Sie die 1. Brillouin-Zone eines ebenen hexagonalen Gitters.}
\label{q:75}

\question{Gitterenergie in einem Ionenkristall.}
\label{q:76}

\question{Röntgenbeugung mit der Drehkristallmethode.}
\label{q:77}

\question{Vergleichen Sie das Einstein- und Debye-Modell der spezifischen Wärme. Welche Annahme ist in BEIDEN Modellen zu einfach?}
\label{q:78}

\newpage
\section{Juli 2017}

\question{Zeichne das Potential eines homöopolaren Moleküls. Beschreiben Sie es qualitativ.}
\label{q:79}

\question{Was bewirkt die Austauschenergie beim H2 Molekül? Wann ist sie groß/klein? Wieso ist sie für die chemische Bindung wichtig?}
\label{q:80}

\question{Was kann man aus dem Rotations-Schwingungsspektrum für molekulare Größen bestimmen?}
\label{q:81}

\question{Diskutieren Sie elektrische Rotations-Schwingungsübergänge und die Entstehung von Schwingungsbanden.}
\label{q:82}

\question{Wie kann experimentell die Dispersionsrelationskurve von Phononen ermittelt werden?}
\label{q:83}

\question{Diskutieren Sie die Wärmekapazität sowohl in der klassischen als auch in der quantenmechanischen Betrachtung.}
\label{q:84}

\question{Was für ein Zusammenhang besteht zwischen der Gitterebene und einem Vektor im reziproken Raum?}
\label{q:85}

\question{Erklären Sie das Auftreten von Energiebandlücken mit Hilfe des Models der fast freien Elektronen.}
\label{q:86}

\question{Beschreiben Sie das Bloch-Theorem eines Elektrons im harmonischen Potential.}
\label{q:87}

\question{Diskutieren Sie die Zustandsdichte in Schwingungsspektren von Festkörpern.}
\label{q:88}

\end{document}